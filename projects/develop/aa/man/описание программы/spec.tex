\documentclass[10pt,a4paper]{report}
\usepackage{cmap}
\usepackage[russian]{babel}
\usepackage{amsmath}
\usepackage{amsfonts}
\usepackage{amssymb}
\usepackage{graphicx}
\usepackage{listings}
\usepackage{hyperref}
\usepackage{float}
\usepackage{fontspec}
\setmainfont{Times New Roman}
\setmonofont{Arial}

\lstset{
	inputencoding=utf8x,
	extendedchars=\true,
	frame=single,
	breaklines=true,
	numbers=left,
	keepspaces = true}

\voffset -24.5mm
\hoffset -5mm
\textwidth 173mm
\textheight 240mm
\oddsidemargin=0mm \evensidemargin=0mm

\author{Абдуллин Азат}
\title{Описание программы}
\begin{document}
	\maketitle
	\renewcommand{\thesection}{\arabic{section}}
	\tableofcontents
	\pagebreak
	
\section{Общие сведения о программе}
Программное обеспечение "Мобильный оконный менеджер XXwm" является оконным менеджером Wayland для операционной системы ArchLinux. Программа написана на языке С. Система сборки:
\begin{itemize}
\item Компилятор GCC
\item Система сборки CMake версии не ниже 3.7
\item Система сборки Make
\end{itemize}
		
\section{Функциональное назначение}
Программное обеспечение предназначено для запуска графических пользовательских приложений в ОС ArchLinux на платформе Rasperry Pi Zero.
Основные функции программы:
\begin{itemize}
\item Запуск системных приложений строки состояния и рабочего стола
\item Запуск пользовательских графических приложений
\item Возможности перемещения и изменения размеров пользовательских окон
\item Обработка нажатия комбинаций клавиш
\end{itemize}

У разработанной программы имеются следующие функциональные ограничения на применение:
\begin{itemize}
\item ОС --- ArchLinux
\end{itemize}
		
\section{Описание логической структуры программы}
Структурно программное обеспечение разделено на две части: подсистема считывания конфигурации и оконный менеджер.
		
\subsection{Подсистема считывания конфигурации}
Подсистема предназначена для считывания конфигурационного файла и предоставления считанной информации оконному менеджеру в соответствующем формате. Подсистема состоит из файлов config.h и config.c.
			
Функции подсистемы:
\begin{itemize}
\item Считывание конфигурационного файла
\item Предоставление считанной информации в соответствующем формате по требованию
\end{itemize}
			
\begin{table}[H]
\caption{Используемые методы и структуры подсистемы считывания конфигурации}
\label{tabular:configSystem}
\begin{center}
\begin{tabular}{| p{0.3\linewidth} | p{0.6\linewidth} |}
\hline
Файл config.h & Содержит объявления функций для считывания конфигурации.\\
\hline
Файл config.с & Содержит определения функций для считывания конфигурации и переменную конфигурации.\\
\hline
Функция \texttt{init\_config} & Функция считывания и инициализации конфигурации. Конфигурационный файл передается в качестве первого аргумента. \\
\hline
Функция \texttt{get\_statusbar} & Функция получения имени исполняемого файла строки состояния в виде строки.\\
\hline
Функция \texttt{get\_desktop} & Функция получения имени исполняемого файла рабочего стола в виде строки.\\
\hline
Структура \texttt{configuration} & Структура, которая хранит в себе считанную конфигурацию.\\
\hline
\end{tabular}
\end{center}
\end{table}
			
\subsection{Подсистема <<Оконный менеджер>>}
Оконный менеджер для запуска графических пользовательских приложений, отображения строки состояния и рабочего стола. Подсистема состоит из одного файла main.c.
		
Функции подсистемы:
\begin{itemize}
\item Запуск пользовательских графических приложений
\item Возможности перемещения и изменения размеров пользовательских окон
\item Обработка нажатия комбинаций клавиш
\end{itemize}

\begin{table}[H]
\caption{Используемые методы и структуры оконного менеджера}
\label{tabular:wmSystem}
\begin{center}
\begin{tabular}{| p{0.3\linewidth} | p{0.6\linewidth} |}
\hline
Файл main.c & Главный файл подсистемы, в котором определены все функции оконного менеджера\\
\hline
Структура \texttt{compositor} & Структура, которая хранит в себе информацию о текущем интерактивном действии.\\
\hline
Функция \texttt{start\_interactive\_action} & Функция, которая начинает интерактивное действие.\\
\hline
Функция \texttt{start\_interactive\_move} & Функция, которая начинает интерактивное действие передвижения окна.\\
\hline
Функция \texttt{start\_interactive\_resize} & Функция, которая начинает интерактивное действие изменения размеров окна.\\
\hline
Функция \texttt{stop\_interactive\_action} & Функция, которая завершает интерактивное действие.\\
\hline
Функция \texttt{relayout} & Функция перерисовывания экрана.\\
\hline
Функция \texttt{output\_resolution} & Функция смены разрешения экрана.\\
\hline
Функция \texttt{view\_created} & Функция обработки отображения нового окна.\\
\hline
Функция \texttt{view\_destroyed} & Функция обработки уничтожения окна.\\
\hline
Функция \texttt{view\_focus} & Функция установки активного окна.\\
\hline
Функция \texttt{view\_request\_move} & Функция запроса перемещения окна.\\
\hline
Функция \texttt{view\_request\_resize} & Функция запроса изменения размера окна.\\
\hline
Функция \texttt{view\_request\_geometry} & Функция запроса изменения параметров отображения окна.\\
\hline
Функция \texttt{keyboard\_key} & Функция обработки клавиш клавиатуры.\\
\hline
Функция \texttt{pointer\_button} & Функция обработки кнопок мыши.\\
\hline
Функция \texttt{pointer\_motion} & Функция обработки перемещения указателя мыши.\\
\hline
Функция \texttt{cb\_log} & Функция логирования.\\
\hline
Функция \texttt{main} & Функция запуска оконного менеджера и системных приложений.\\
\hline
\end{tabular}
\end{center}
\end{table}
		
\section{Используемые технические средства}
При работе данной программы используется персональный компьютер с ОС ArchLinux. Возможен так же запуск на платформе Raspberry Pi Zero.
	
\section{Вызов и загрузка}
Для использования данной программы необходимо установить все файлы программы в соответствии с документом <<Руководство программиста>> и выбрать программу в экранном менеджере при входе в систему.
	
\section{Входные данные}
Входными данными, формируемыми пользователем, является конфигурационный файл.
		
Входными данными для изделия является конфигурационный файл, сформированный пользователем.
		
\section{Выходные данные}
Выходными данными для изделия является запущенный оконный менеджер и системные приложения рабочего стола и строки состояния
\end{document}