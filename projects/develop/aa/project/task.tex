\documentclass[11pt,a4paper]{extarticle}
\usepackage[left=3cm, right=1cm, top=2cm, bottom=2cm]{geometry}
\usepackage[utf8]{inputenc}
\usepackage[russian]{babel}
\usepackage[OT1]{fontenc}
\usepackage{amsmath}
\usepackage{amsfonts}
\usepackage{amssymb}
\usepackage{graphicx}
\usepackage{listings}
\graphicspath{{image/}}
\bibliographystyle{plain}
\begin{document}
\large{\textbf{Техническое задание}}

Необходимо создать мобильный оконный менеджер (ОМ). Конечной целью работы является запуск созданного ОМ на платформе Raspberry Pi Zero. Из-за малой мощности необходимо, чтобы ОМ работал на протоколе Wayland.

Wayland --- протокол для организации графического сервера в Linux и других UNIX-подобных операционных системах, а также его библиотечная реализация в Си. В роли клиента может выступать пользовательское приложение, X сервер или другой дисплейный сервер. 
\begin{itemize}
\item Цель: радикально упростить графическую среду Linux по сравнению с X Window System.
\item Использует Unix Domain Sockets, сетевой прозрачности нет. 
\item Главным образом использует DRI (Direct Rendering Infrastructure) --- интерфейс доступа к видеоаппаратуре. 
\item Устройства ввода-вывода управляются полностью из ядра. 
\item Распределение буфера и отрисовка полностью на стороне клиента.
\end{itemize}

\begin{figure}[h!]
\center
\includegraphics[width=\linewidth]{image/wayland-architecture}
\caption{Архитектура Wayland}
\label{img:wayland-architecture}
\end{figure}

Как показано на рисунке \ref{img:wayland-architecture}, ключевым понятием в архитектуре Wayland является композитор. Композитор --- это дисплейный сервер, который взаимодействуют с пользовательскими устройствами ввода-вывода, с железом, управляет потоком данных клиентских программ. В конечном счете композитор работает с буферами вывода всех отображаемых окон, определяет как эти буфера будут располагаться в буфере вывода дисплея. В X Window System функциональность композитора была вынесена в реализацию сервера (X Server), ОМ ничего об этом не знал. Однако, в ОМ для Wayland реализация композитора это один из основных этапов разработки. Так как реализация собственного композитора является очень объемной и сложной задачей (даже простейшие композиторы занимают ~ 10-15к строк кода), было решено при реализации ОМ использовать какую-либо библиотеку, содержащую в себе композитор. Таким образом, задача созданного ОМ будет заключаться в управлении пользовательскими окнами.

Так же было сказано, что конечной целью разработки является запуск ОМ на Raspberry Pi Zero. Однако, RPi имеет свою проприетарную графику, которая не поддерживается существующими композиторами и оконными менеджерами. Как было указано выше, Wayland использует DRI (для обеспечения аппаратного ускорения с использованием Mesa3D). Таким образом, необходимо настроить RPi для представления его графического устройства в виде DRM (Direct Rendering Manager). Для этого необходимо настроить драйвер VC4 для RPi. VC4 --- это драйвер, включенный в пакет Mesa3D, который позволяет представить проприетарное графическое устройство RPi в виде стандартного DRM устройства.

Мобильный оконный менеджер должен так же иметь два встроенных системных приложения:
\begin{itemize}
\item Строка состояния, которая отображает информацию об устройстве (текущее время, уровень заряда, уровень сигнала и т.д.). Строка состояния всегда отображается в верхней части экрана.
\item Рабочий стол, на котором расположены иконки запуска приложений, установленных в системе. Рабочий стол отображается на всю часть экрана, не занятую строкой состояния.
\end{itemize}
Данные системные приложения не являются непосредственной частью ОМ. ОМ автоматически запускает их при своем запуске, и запоминает их PID для соответствующего их отображения. В ОМ должна присутствовать возможность конфигурирования. В частности: возможность указать путь к приложениям строки состояния и рабочего стола.

При запуске несистемных приложений ОМ скрывает рабочий стол, выводит запущенное приложение на передний план и делает его активным. Так же ОМ должен поддерживать три типа окон:
\begin{itemize}
\item обычные окна приложений, которые отображаются во весь экран
\item всплывающие уведомления (например, контекстное меню при нажатии правой кнопки мыши), которые отображаются в соответствии со своими размерами в указанной точке. Например, при нажатии на правую кнопку мыши открывается контекстное меню в точке нажатия с необходимыми размерами
\item окна меню (например стандартные меню типа "Файл" и т.д. в верхней части приложений). Должны запускаться в соответствии со своими размерами и отрисовываться начиная с нажатой кнопки.
\end{itemize}

Таким образом, задачу можно разделить на две практические независимые части:
\begin{enumerate}
\item Настройка ОС RPi для представления графического устройства в виде стандартного DRM
\item Разработка мобильного ОМ для Wayland:
	\begin{itemize}
	\item использование библиотеки-композитора
	\item поддержка системных приложений
	\item конфигурируемость ОМ
	\item поддержка и соответствующее отображение трех типов окон (обычное окно, всплывающее окно, окно меню)
	\end{itemize}
\end{enumerate}

\end{document}