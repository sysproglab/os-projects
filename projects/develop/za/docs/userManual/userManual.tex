\documentclass[14pt,a4paper]{report}
\usepackage{cmap}
\usepackage[russian]{babel}
\usepackage{amsmath}
\usepackage{amsfonts}
\usepackage{amssymb}
\usepackage{graphicx}
\usepackage{listings}
\usepackage{hyperref}
\usepackage{fontspec}
\setmainfont{SFNS Display}
\setmonofont{SFNS Display}

\lstset{
	inputencoding=utf8x,
	extendedchars=\true,
	frame=single,
	breaklines=true,
	numbers=left,
	keepspaces = true}

\voffset -24.5mm
\hoffset -5mm
\textwidth 173mm
\textheight 240mm
\oddsidemargin=0mm \evensidemargin=0mm

\author{Зорин А.Г.}
\title{Руководство пользователя}
\begin{document}
\maketitle
\renewcommand{\thesection}{\arabic{section}}
\section{Введение}
Данный демон предназначен для активации подключенного SIM-модуля, а также для обеспечения получения базовой информации, такой как информация об операторе. Данный демон не зависит от ОС и платформы реализации. Для запуска демона в системе должны быть установлены следующие пакеты:
\begin{itemize}
\item oFono
\item Qt
\item GLib
\end{itemize}
	
\section{Подготовка к работе}
Самый первый шаг, перед запуском демона --- установка пакета oFono. Осуществляется установка следующим образом:
\begin{verbatim}
git clone https://git.ejiek.com/KSPT/ofono.git
cd ofono/
git checkout kspt
./bootstrap-configure
make
sudo make install
sudo install -Dm755 "./src/ofono.conf" "/etc/dbus-1/system.d/ofono.conf"
sudo install -Dm755 "./src/ofono.service" "/usr/lib/systemd/system/ofono.service"
\end{verbatim}

В результате выполнения данных команд, в системе должен появиться сервис, который называется \textit{ofono}. Следующий этап настройки oFono - создание файла \textit{ofono.rules} в каталоге \textit{/etc/udev/rules.d/}. В созданный фал необходимо записать одну строчку:
\begin{verbatim}
KERNEL=="ttyUSB0", ENV{OFONO_DRIVER}="sim900"
\end{verbatim}

Однако, файлы устройства могут меняться, в зависимости от того, каким образом подключено устройство. Если подключение через USB --- \textit{/dev/ttyUSB0}, а если через UART --- \textit{/dev/ttyAMA0}.

Следующим этапом скажем нашей системе, что сервис ofono необходимо включать вмете с системой. Для этого необходимо выполнить следующую команду:
\begin{verbatim}
sudo systemctl enable ofono
\end{verbatim}
Для того, чтобы настройки вступили в силу, необходимо перезагрузить ОС. 

\section{Описание работы}
После проведения всех настроек, для того, чтобы запустить демон, его необходимо собрать. Система сборки демона --- CMake. Из этого следует, что сборку можно выполнить следующими командами:
\begin{verbatim}
mkdir build
cd build
cmake ..
make
\end{verbatim}
Сборка демона выполняется быстро. По ее окончании, необходимо запустить исполняемый файл. С этих пор, демон будет работать в системе и отслеживать изменения.

\section{Аварийные ситуации}
При неправильной сборке или установке oFono, может возникнуть такая ситуация, что сервис в системе не был создан. Вероятнее всего, не были выполнены последние две строчки из руководства по установке.

В том случае, если в демоне произошла какая-то ошибка, все будет записано в системный лог-файл.
\end{document}
