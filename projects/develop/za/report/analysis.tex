\section{Анализ способов общения через D-Bus}
\subsection{QtDbus}
Qt --- это кросплатформенный инструментарий разработки программного обеспечения (ПО) на языке программирования \textit{C++}~\cite{Qt}. Qt позваляет запускать различные приложения, написанные с его помощью, на различных ОС, без необходимости переписывать исходный код. Одна из основных отличительных особенностей Qt - использование \textit{Meta Object Compiler (MOC)}. MOC - система предварительной обработки исходного кода. Она позволяет во много раз увеличить мощь библиотек вводя таки понятия, как \textit{слоты} и \textit{сигналы}. 

Qt позволяет создавать собственные плагины и размещать их непосредственно в панели визуального редактора. Также существует возможность расширения привычной функциональности виджетов, связанной с размещением их на экране, отображением, перерисовкой при изменении размеров окна.

Одним из весомых преимуществ проекта Qt является наличие качественной документации. Статьи документации снабжены большим количеством примеров. Исходный код самой библиотеки хорошо форматирован, подробно комментирован и легко читается, что также упрощает изучение Qt~\cite{QtDoc}.

Помимо <<чистого>> Qt, для реализации графического интерфейса можно использовать связку \textit{Qt + QML}. QML представляет из себя декларативный язык программирования, основанный на \textit{JavaScript}, предназначенный для создания дизайна приложений. QML документ выглядит как дерево элеиентов. Сам QML элемент представляет из себя совокупность блоков:
\begin{itemize}
\item Графических
	\begin{itemize}
	\item Rectangle
	\item Image и т.д.
	\end{itemize}
\item Поведенческих
	\begin{itemize}
	\item State
	\item Transition
	\item Animation и т.д.
	\end{itemize}
\end{itemize}

Qt обеспечивает возможностью работать с D-Bus через собственный модуль, который называется QtDBus. Данный модуль полностью инкапсулирует низкоуровневую концепцию обмена сообщений в более простую - объектно ориентированную модель. Для работы с данном модулем существует огромное количество классов, каждый из которых хорошо задокументирован.

\subsection{GLib}
GLib --- набор из низкоуровневых системных библиотек, написанных на языке программирования \textit{C} и разрабатываемых, в основном, \textit{GNOME}~\cite{gnome}. Исходные коды GLib были отделены от GTK+ и могут быть использованы ПО отличным от \textit{GNOME}. GLib распространяется под лицензией GNU GPL и исходные коды находятся на \textit{github}~\cite{glibSources}.

GLib предоставляет такие структуры данных, как:
\begin{itemize}
\item Одно- и двусвязные списки
\item Хэш-таблицы
\item Динамические массивы
\item Динамическиие строки
\item Сбалансированные двоичные деревья и т.д.
\end{itemize}



